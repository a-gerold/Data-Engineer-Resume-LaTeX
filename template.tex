%%%%%%%%%%%%%%%%%%%%%%%%%%%%%%%%%%%%%%%%%
% Twenty Seconds Resume/CV
% LaTeX Template
% Version 1.0 (14/7/16)
%
% Original author:
% Carmine Spagnuolo (cspagnuolo@unisa.it) with major modifications by 
% Vel (vel@LaTeXTemplates.com) and Harsh (harsh.gadgil@gmail.com)
%
% License:
% The MIT License (see included LICENSE file)
%
%%%%%%%%%%%%%%%%%%%%%%%%%%%%%%%%%%%%%%%%%

%----------------------------------------------------------------------------------------
%	PACKAGES AND OTHER DOCUMENT CONFIGURATIONS
%----------------------------------------------------------------------------------------

\documentclass[a4paper]{twentysecondcv} % a4paper for A4

% Command for printing skill overview bubbles
\newcommand\skills{ 
~
	\smartdiagram[bubble diagram]{
        \textbf{Software}\\\textbf{Entwicklung},
        \textbf{~~~~~Angular~~~~~},
        \textbf{~~~~~~~~.NET~~~~~~~~~},
        \textbf{~~OOP~~}\\\textbf{~~OOD~~},
        \textbf{~~~~~~REST~~~~~~},
        \textbf{~~Datenbanken~~},
        \textbf{Automotive}
    }
}

% Programming skill bars
\programming{
	{C $\textbullet$ C++ $\textbullet$ Java / 1},
	{SQL $\textbullet$ CSS $\textbullet$  HTML / 3.5},
	{C\# $\textbullet$ TypeScript $\textbullet$ JavaScript / 5}}

\otherskills{
	{Ubuntu $\textbullet$ LXD $\textbullet$ Samba / 4},
	{Windows Server $\textbullet$ Hyper-V $\textbullet$  IIS / 5},
	{GIT $\textbullet$ Jira $\textbullet$ GitLab / 5},
	{Netzwerke $\textbullet$ Routing $\textbullet$ DNS / 5}}

% Projects text
\education{
\textbf{Master of Science (Note: 1,8)
\\Elektro- und Informationstechnik} \\
Nano- und Quantenelektronik \\
Technische Universität München \\
2013 - 2016

\textbf{Bachelor of Engineering (Note: 1,6)
\\Elektro- und Informationstechnik} \\
Kommunikationstechnik \\
Hochschule München \\
2010 - 2013
}

%----------------------------------------------------------------------------------------
%	 PERSONAL INFORMATION
%----------------------------------------------------------------------------------------
% If you don't need one or more of the below, just remove the content leaving the command, e.g. \cvnumberphone{}

\cvname{Andreas GEROLD} % Your name
\cvjobtitle{Entwickler und Ingenieur} % Job
% title/career

\cvlinkedin{/in/a-gerold}
\cvnumberphone{0160 985 23 430} % Phone number
\cvmail{a.gerold@gmail.com} % Email address

%----------------------------------------------------------------------------------------

\begin{document}

\makeprofile % Print the sidebar

%----------------------------------------------------------------------------------------
%	 EXPERIENCE
%----------------------------------------------------------------------------------------

\section{Erfahrung}

\begin{twenty} % Environment for a list with descriptions
\twentyitem
    	{Sep 2016 -}
		{heute}
        {Softwareentwickler}
        {\href{https://www.magna.com/de}{Magna Steyr Engineering Deutschland}}
        {}
        {\begin{itemize}
        \item Mitwirken beim Aufbau eines fünfköpfigen SCRUM-Teams
        \item Entwicklung eines RESTful, Data-Warehouse-artigen, queryablen Backends mit OASIS Standard (ODATA) unter Verwendung der .NET WebApi (C\#), EntityFramework, OWIN Pipeline, Dependency Injection und WebSockets (SignalR)
        \item Anbindung verschiedener Datenbanken an das Backend
        \item Entwicklung von Angular Komponenten, Direktiven und Services inklusive Frontends mit FlexBox und Material Design
        \item Schärfen von Userstories mit dem Productowner, Konzeption von Funktionen mit dem Productowner
        \item Konfiguration und Wartung des unter Linux selbst gehosteten CI / CD / Repository Portals GITLab
        \end{itemize}}
        \\
	\twentyitem
    	{Dez 2014 -}
		{Sep 2016}
        {Entwicklungsingenieur}
        {\href{https://www.magna.com/de}{Magna Steyr Engineering Deutschland}}
        {}
        {
        {\begin{itemize}
        \item Entwurf und Implementierung einer MS SQL Datenbank zur \\ 
        Speicherung der Prozess- und Projektdaten
        \item Implementierung verschiedener Parser mit C\#
        \item Testen von Kunden- und Systemfunktionen von Fahrzeug-Prototypen
    \end{itemize}}
        }
    \\   
    \twentyitem
   		{Mär 2013 -}
		{Dez 2014}
        {Entwicklungsingenieur 50\%}
        {\href{https://www.magna.com/de}{Magna Steyr Engineering Deutschland}}
        {}
        {
        {\begin{itemize}
        \item Teilzeitstelle parallel zum Studium
        \item Tausch und Flashen von Steuergeräten in Fahrzeug-Prototypen
        \item Fehleranalyse auf BUS-Ebene, hauptsächlich CAN-BUS
    \end{itemize}}
        }
     \\
     \twentyitem
   		{Mär 2010 -}
		{Sep 2012}
        {Werkstudent (20h/Woche)}
        {\href{https://www.magna.com/de}{Magna Steyr Engineering Deutschland}}
        {}
        {
        \begin{itemize}
        \item Verschiedene Projeke in Zusammenhang mit der Systemintegration in der Fahrzeugentwicklung
    \end{itemize}
    	}
        
	%\twentyitem{<dates>}{<title>}{<location>}{<description>}
\end{twenty}

%----------------------------------------------------------------------------------------
%	 RESEARCH
%----------------------------------------------------------------------------------------
\section{Abschlussarbeiten}
\begin{twenty}
	\twentyitem
    	{Sep 2015 -}
		{Feb 2016}
        {Masterarbeit (Note: 1,3)}
        {\href{https://www.tum.de/}{Technische Universität München}}
        {}
        {
       	\textbf{Titel}: Datenanalyse und Modellgenerierung für das Sichtfeld von Fahrzeug RADAR und LIDAR Sensoren
        {\begin{itemize}
        \item Analyse von realen Sensor Messdaten mit MatLab zur Ableitung von Parametern wie Sichtweite, Öffnungswinkel, etc.
        \item Modellierung von Störungen für den Fahrsimulator mit C++
        \item Parametrierung der Störmodelle durch die abgeleiteten Größen aus den realen Messdaten
        \item Verifikation der Störmodelle durch Anwendung auf die Ground-Truth und Vergleich mit den realen Daten
        \item \textbf{Tools}: MatLab, Visual Studio 2015 (C++) \vspace{2mm}
		\end{itemize}}
        }
    \twentyitem
    {Sep 2012 -}
    {Feb 2016}
    {Bachelorarbeit (Note: 1,3)}
    {\href{https://www.hm.edu/}{Hochschule München}}
    {}
    {
    	\textbf{Titel}: Automatisierung von Tests zur Hochvoltsicherheit von Elektro- und Hybridfahrzeugen
    	{\begin{itemize}
    			\item Bestandsaufnahme und Bewertung der existierenden Testfälle
    			\item Identifikation von nutzbaren Schnittstellen zu den Fahrzeugen
    			\item Implementierung ausgewählter Testfälle mit der Software Tracetronik ECU-Test
    			\item Verifikation und Bugfixing der Implementierung durch Testdurchführung am Fahrzeug
    			\item \textbf{Tools}: \href{https://www.tracetronic.de/produkte/ecu-test/}{Tracetronik ECU-Test} \vspace{2mm}
    	\end{itemize}}
    }
\end{twenty}

\section{Interessen}
OpenSource, Netzwerktechnik, Mountainbiken, Klettern, Skitouren

\end{document} 
